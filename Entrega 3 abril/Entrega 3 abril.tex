\documentclass[fleqn]{article}


\usepackage{mathexam}
\usepackage{amsmath}
\usepackage{amsfonts}
\usepackage{graphicx}
\usepackage{systeme}
\usepackage{microtype}
\usepackage{multirow}
\usepackage{pgfplots}
\usepackage{listings}
\usepackage{tikz}
\usepackage{dsfont} %Numeros reales, naturales...

\graphicspath{{images/}}
\newcommand*{\QED}{\hfill\ensuremath{\square}}

%Estructura de ecuaciones
%\setlength{\textwidth}{17cm} \setlength{\oddsidemargin}{-10mm}
%\setlength{\textheight}{25cm} \setlength{\topmargin}{-2cm}

\author{David García Curbelo}
\title{Probabilidad}

\pagestyle{empty}


\def\R{$\mathds{R}$}
\def\sup{$^2$}

\begin{document}


    \begin{center}
        {\large\bf{Ejercicios para entregar. Ecuaciones diferenciales, 3/04}} \\
        %\bf{David García Curbelo}\\
        
    \end{center}
    
    {\bf{Ejercicio 3.2.3.}} \textit{¿Qué ocurre si repetimos el proceso del ejemplo anterior a las funciones $P(x,y)=e^x$ y $Q(x,y)=e^y + x $?}\\
    
    {\bf{Solución}}  Consideramos $P(x,y)=e^x$ y $Q(x,y)=e^y + x $. Ambas funciones están definidas en \R\sup, el cual sabemos que es un dominio estrellado.
    Veamos si se cumple la condición de exactitud.
    $$\frac{\partial P}{\partial y}=\frac{\partial Q}{\partial x}$$
    Como $\frac{\partial P}{\partial y}=0$ y $\frac{\partial Q}{\partial x}=1$ observamos que no se cumple la condición de exactitud mencionada, 
    y por tanto no se cumplirá la Proposición 3.2.1., es decir, no se cumple la condición
    $$\frac{\partial U}{\partial x}(x,y)=P(x,y), \frac{\partial U}{\partial y}(x,y)=Q(x,y)$$
    y por ello podemos afirmar que no existe $U$ bajo las condiciones de $P(x,y)$ y $Q(x,y)$ enunciadas.\\
    
    Otra forma de Solucionar el problema es realizando las cuentas del ejemplo anterior y comprobando que se llega a contradicción. Para ello debemos de suponer
    que la condición de exactitud se cumple, con lo que procedemos a calcular $U$ mediante integración.
    $$U(x,y)=\int P(x,y) dx + f(y) = \int e^x dx + f(y) = e^x + f(y)$$
    $$U(x,y)=\int Q(x,y) dy + f(x) = \int (e^y + x)dy + f(x) = e^y + xy + f(x)$$
    Podemos ver fácilmente que $f(x)=e^x$, pero para $f(y)$ no queda tan claro ya que tendríamos que tomar $f(y)=e^y + xy$, la cual no depende únicamente de $y$,
    luego llegamos a una contradicción. Por ello, no existe un potencial para las condiciones de $P(x,y)$ y $Q(x,y)$ enunciadas.\QED

    \newpage
    {\bf{Ejercicio 3.2.4.}} \textit{Consideramos $P(x,y)=e^x + 2y$ y $Q(x,y)=2x + \cos(y) $. Determina U como indica el teorema 3.2.1, es decir, calcula
    $$U(x,y)=x\int _{0}^{1} P(\lambda x, \lambda y) d\lambda + y\int _{0}^{1} Q(\lambda x, \lambda y) d\lambda $$
    Calcula las derivadas parciales empleando el Lema 3.2.1.}\\

    {\bf{Solución}} Consideramos $P(x,y)=e^x + 2y$ y $Q(x,y)=2x + \cos(y) $. Ambas funciones están definidas en \R\sup, el cual sabemos que es un dominio 
    estrellado. Sabemos además por el ejemplo anterior que estas dos funciones cumplen la condición de exactitud. Procedamos al cálculo de U mediante 
    la integral
    $$\int _{0}^{1} xP(\lambda x, \lambda y) + yQ(\lambda x, \lambda y) d\lambda$$
    $$\int _{0}^{1} xe^{x\lambda} d\lambda +\int _{0}^{1} 2xy\lambda d\lambda + \int _{0}^{1} 2xy\lambda d\lambda + \int _{0}^{1} y\cos(y\lambda) d\lambda$$
    $$x\int _{0}^{1} e^{x\lambda} d\lambda + 4xy\int _{0}^{1} \lambda d\lambda + y\int _{0}^{1} \cos(y\lambda) d\lambda$$
    $$[x \frac{1}{x} e^{x\lambda}+2xy\lambda^2+y\frac{1}{y} \sin(y\lambda)]_{0}^{1} = e^x + 2xy + \sin(y) - 1$$
    El cual vemos que se trata de la solución que buscábamos, con la constante $C=-1$,
    $$U(x,y)= e^x + 2xy + \sin(y) - 1$$
    
    Para el segundo apartado, calcularemos las derivadas parciales de $U(x,y)$ y veremos que se verifica $\frac{\partial U(x,y)}{\partial x}=P(x,y)$,
    $\frac{\partial U(x,y)}{\partial y}=Q(x,y)$. Aplicamos para ello el Lema 3.2.1. por el cual partiendo de
    $$U(x,y)=\int _{0}^{1} xP(\lambda x, \lambda y) + yQ(\lambda x, \lambda y) d\lambda$$
    y aplicando el lema obtenemos
    $$\frac{\partial U(x,y)}{\partial x}=\int _{0}^{1} \frac{\partial}{\partial x}(xP(\lambda x, \lambda y) + yQ(\lambda x, \lambda y)) d\lambda$$
    $$\frac{\partial U(x,y)}{\partial x}=\int _{0}^{1} P(\lambda x, \lambda y) d\lambda + x\int _{0}^{1} \lambda \frac{\partial}{\partial x} P(\lambda x, \lambda y) d\lambda + y\int _{0}^{1} \lambda \frac{\partial}{\partial x} Q(\lambda x, \lambda y)) d\lambda$$
    
    Sustituimos a continuación las respectivas funciones P y Q. Como sabemos que se cumple la condición de exactitud,
    tenemos
    $$\frac{\partial U(x,y)}{\partial x}=\int _{0}^{1} P(\lambda x, \lambda y) d\lambda + x\int _{0}^{1} \lambda \frac{\partial}{\partial x} P(\lambda x, \lambda y) d\lambda + y\int _{0}^{1} \lambda \frac{\partial}{\partial y} P(\lambda x, \lambda y) d\lambda$$
    $$\frac{\partial U(x,y)}{\partial x}=\int _{0}^{1} (e^{x\lambda} + 2y\lambda) d\lambda + x\int _{0}^{1} \lambda \frac{\partial}{\partial x} (e^{x\lambda} + 2y\lambda) d\lambda + y\int _{0}^{1} \lambda \frac{\partial}{\partial y} (e^{x\lambda} + 2y\lambda) d\lambda$$

    Ahora, mediante $\frac{dP}{\lambda d}(\lambda x, \lambda y)=x \frac{\partial}{\partial x} P(\lambda x, \lambda y) + y \frac{\partial}{\partial y} P(\lambda x, \lambda y)$ obtenemos
    $$\frac{\partial U(x,y)}{\partial x}=\int _{0}^{1} (e^{x\lambda} + 2y\lambda) d\lambda + \int _{0}^{1} \lambda \frac{d}{\lambda d}(e^{x\lambda} + 2y\lambda) d\lambda$$
    $$\frac{\partial U(x,y)}{\partial x}=\left(\frac{e^x - 1}{x} + y\right) + \left(e^x + \frac{1 - e^x}{x} + y\right)$$
    
    $$\frac{\partial U(x,y)}{\partial x}=e^x + 2y = P(x,y)$$

    Como queríamos probar. Para $Q(x,y)$ el proceso sería el mismo. \QED

\end{document}