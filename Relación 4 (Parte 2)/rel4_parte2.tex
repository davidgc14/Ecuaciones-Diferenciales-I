\documentclass[fleqn]{article}

%\pgfplotsset{compat=1.17}

\usepackage{mathexam}
\usepackage{amsmath}
\usepackage{amsfonts}
\usepackage{graphicx}
\usepackage{systeme}
\usepackage{microtype}
\usepackage{multirow}
\usepackage{pgfplots}
\usepackage{listings}
\usepackage{tikz}
\usepackage{dsfont} %Numeros reales, naturales...
\usepackage{cancel}

%\graphicspath{{images/}}
\newcommand*{\QED}{\hfill\ensuremath{\square}}

%Estructura de ecuaciones
\setlength{\textwidth}{15cm} \setlength{\oddsidemargin}{5mm}
\setlength{\textheight}{23cm} \setlength{\topmargin}{-1cm}



\author{David García Curbelo}
\title{Ecuaciones}

\pagestyle{empty}


\def\R{\mathds{R}}
\def\Z{\mathds{Z}}
\def\N{\mathds{N}}

\def\sup{$^2$}

\def\next{\quad \Rightarrow \quad}

\begin{document}

    \setcounter{page}{1}
    \pagestyle{plain}
    \markright{Relación 4 Ecuaciones Diferenciales}

    \begin{center}
        {\large\bf{Relación 4 Ecuaciones diferenciales}} \\
        \bf{Parte 2}\\
        
    \end{center}

    \textbf{Ejercicio 1. } \\

    Nos piden encontrar la solución al sistema dado por 
    \begin{equation*}
        \left.
        \begin{aligned}
            x' + ty &= -1\\
            y' + x' &= -1\\
        \end{aligned}
        \right\}\quad \Rightarrow \quad
        \left.
        \begin{aligned}
            x' &= -ty -1\\
            y' &= ty + 1 - 1\\
        \end{aligned}
        \right\}\quad \Rightarrow \quad
        \left.
        \begin{aligned}
            x' &= -ty -1\\
            y' &= ty\\
        \end{aligned}
        \right\} 
    \end{equation*}
    Dicho sistema podemos expresar de forma $X'(t)=A(t)X(t) + B(t)$, donde vemos que cada uno de los elementos viene dado por

    \begin{equation*}
        X'(t) = 
        \begin{pmatrix}
            x' \\
            y' \\
        \end{pmatrix}\\
        A(t) = 
        \begin{pmatrix}
            0 & -t  \\
            0 & t  \\
        \end{pmatrix}\\
        X(t) = 
        \begin{pmatrix}
            x \\
            y \\
        \end{pmatrix}\\
        B(t) = 
        \begin{pmatrix}
            -1 \\
            0 \\
        \end{pmatrix}\\
    \end{equation*}
    Nos encontramos ante un sistema de ecuaciones diferenciales lineales de primer orden. Podemos ver que la segunda ecuación sólo tiene coeficientes
    en $y$,  ya que viene dada por $y' = ty$, la cual sabemos que tiene por solución
    $$y(t) = K\cdot e^{\frac{t^2}{2}}, \quad K\in \R$$
    Sustituyendo en la primera ecuación, la cual viene dada por $x' = -ty -1$, que sustituyendo la solución obtenida de la ecuación anterior obtenemos
    $$x'(t) = -tKe^{\frac{t^2}{2}} - 1$$
    Que integrando en ambos lados resulta
    $$x(t) = - Ke^{\frac{t^2}{2}} - t + C, \quad C\in \R$$
    y obtenemos así la solución general del sistema planteado, en función de un parametro $K\in\R$, el cual variará em función de la condición inicial.
    $$X(t) = \begin{pmatrix}
        Ke^{\frac{t^2}{2}} - t + C\\
        Ke^{\frac{t^2}{2}} \\
    \end{pmatrix}, \quad K\in \R$$
    
    Y ahora, para el cálculo de una matriz solución particular, bastará con sustituir los parámetros que tenemos para obtener dos soluciones distintas del problema
    inicial. Tomemos por ello
    \begin{equation*}
        x_1(t) = 
        \begin{pmatrix}
            -t \\
            0 \\
        \end{pmatrix}\\
        x_2(t) = 
        \begin{pmatrix}
            e^{\frac{t^2}{2}} - t \\
            e^{\frac{t^2}{2}} \\
        \end{pmatrix}
    \end{equation*}
    Donde hemos tomado para la primera solución $K=C=0$, y para la segunda $K=1, \thinspace C=0 $, y por ello podemos construir su matriz solución dada por:
    $$
        \Phi (t) = 
        \begin{pmatrix}
            -t & -e^{\frac{t^2}{2}} - t \\
            0 & e^{\frac{t^2}{2}} \\           
        \end{pmatrix}
    $$\\ \\

    \newpage 

    \textbf{Ejercicio 2. } \\

    Como tenemos la matriz $\Phi$, supongamos que es matriz solución de un sistema dado $X'(t) = A(t) X(t)$. Por ser matriz solución (y ser clase $\mathcal{C}^1$ y por tanto
    ser derivable), sabemos que tiene que cumplirse $\Phi' = A\Phi$. Como el determinante de la matriz $\Phi$ es no nulo $\forall t \in I\subset \R$, podemos afirmar que dicha matriz 
    tiene inversa, y que por lo tanto la matriz $A$ puede representarse de la siguiente manera
    $$A=\Phi' \Phi^{-1}$$
    Por ello vemos que hemos determinado la matriz $A$ en función de $\Phi$, para la cual, el problema $X'(t) = A(t) X(t)$ tiene por matriz solución la matriz $\Phi$, y que  por lo tanto (por tener
    determinante no nulo) podemos afirmar que se trata de una matriz fundamental.\\ \\

    \textbf{Ejercicio 3. } \\

    Supondremos durante la demostración que ambas matrices son clase $\mathcal{C}^1$. Si queremos probar la existencia de un $C \in \R$ tal que se cumpla $\Phi = \Psi \cdot C$, será lo mismo probar que $C = \Phi \Psi^{-1}$ 
    (podemos afirmar que tiene inversa ya que tiene determinante no nulo) y por tanto 
    $$(\Psi^{-1}\Phi)' = 0$$
    Por tanto veamos si, siendo ambas matrices fundamentales, estudiemos el valor de la derivada. 
    $$(\Psi^{-1}\Phi)' = \Psi^{-1} \Phi'  + (\Psi^{-1})' \Phi $$
    Estudiemos a parte el valor de $(\Psi^{-1})'$. Para ello consideremos $\Psi \Psi^{-1} = 1_n$, y por tanto la derivada ha de ser nula, 
    $$(\Psi^{-1}\Psi)' = \Psi' \Psi^{-1} + \Psi (\Psi^{-1})'= 0 \next (\Psi^{-1})' = - \Psi^{-1}\Psi' \Psi^{-1}$$
    Podemos considerar inversas puesto que, al tratarse de matrices fundamentales tienen determinante no nulo. Retomemos por tanto la ecuación anterior, 
    $$(\Psi^{-1}\Phi)' = \Psi^{-1}\Phi' + (\Psi^{-1})' \Phi  = \Psi^{-1} \Phi' - \Psi^{-1}\Psi' \Psi^{-1} \Phi  $$
    Como ambas matrices son fundamentales, son por ello solución de $X'(t) = A(t) X(t)$ y por tanto la cumplen $\Phi' = A\Phi$ y $\Psi' = A\Psi$. Sustituimos y obtenemos
    $$(\Psi^{-1}\Phi)' = \Psi^{-1} A(t)\Phi - \Psi^{-1}A(t) \cancel{\Psi \Psi^{-1}} \Phi = \Psi^{-1} A(t)\Phi - \Psi^{-1} A(t)\Phi = 0$$
    $$(\Psi^{-1}\Phi)' = 0$$
    Como queríamos probar.\\ \\

    \textbf{Ejercicio 4. } \\

    Repetimos la demostración hecha en el apartado anterior. Como nuestra matriz es clase $\mathcal{C}^1$ con determinante no nulo, podemos afirmar que tiene inversa. También podemos afirmar
    la obviedad $\Phi \Phi^{-1} = 1_n$, y por tanto si derivamos ambos lados de la igualdad obtenemos
    $$(\Phi^{-1}\Phi)' = \Phi' \Phi^{-1} + \Phi (\Phi^{-1})'= 0 \next - \Phi' \Phi^{-1} = \Phi (\Phi^{-1})'$$
    Como hemos dicho que tiene inversa, basta despejar $(\Phi^{-1})'$ y obtenemos el resultado que buscábamos
    $$(\Phi^{-1})' = - \Phi^{-1} \Phi' \Phi^{-1} $$\\ \\

    \textbf{Ejercicio 5. } \\
    El enunciado es falso. Consideremos un sistema homogéneo $X'(t) = A X(t)$ (con la matriz $A$ con coeficientes constantes) tal que $\Phi$ sea una matriz fundamental (clase $\mathcal{C}^1$) del sistema homogeneo
    dado. Por tanto sabemos que $\Phi$ es solución del sistema, por lo que obtenemos 
    $$\Phi' = A \Phi$$
    Como hemos supuesto que la matriz $A$ tiene coeficientes constantes, al derivar en ambos lados de la igualdad obtenemos 
    $$\Phi'' = A \Phi'$$
    Luego $\Phi'$ también es solución del sistema homogéneo $X'(t) = A X(t)$.
    
    
    
    Toda matriz fundamental tiene determinante no nulo, por lo que basta tomar una matriz $\Phi$ tal que $det(\Phi) = C, \thinspace C \in \R-\{0\}$,
    con lo que tenemos una matriz con determinante no nulo para todo $t$ en su dominio de definición\\ \\

    \textbf{Ejercicio 6. } \\

    Para ver si son solución de manera simultánea, podemos considerar la matriz solución 
    $$\Phi (t) = \begin{pmatrix}
        1 & \sin(t)\\
        \sin(t) & 1\\
    \end{pmatrix}$$
    Y comprobar si existe una matriz $A(t)$ para que la matriz $\Phi$ sea solución del sistema homogéneo $X'(t) = A(t) X(t)$. Si suponemos que sea matriz solución,
    entonces tiene que cumplirse
    $$\Phi' (t) = A(t) \Phi(t)$$
    $$
    \begin{pmatrix}
        0 & \cos(t)\\
        \cos(t) & 0\\
    \end{pmatrix} 
    = A(t)
    \begin{pmatrix}
        1 & \sin(t)\\
        \sin(t) & 1\\
    \end{pmatrix}
    $$
    Si estudiamos el determinante de la matriz $\Phi$ vemos que $det(\Phi)=1-\sin^2(t)=\cos^2(t)$, el cual vemos que es no nulo para todo 
    $t\neq K\pi + \frac{\pi}{2}, \thinspace K\in\Z$. Por lo tanto, considerando dicho determinante no nulo, podemos afirmar que tiene inversa, y por tanto 
    podemos despejar la matriz $A(t)$ de la siguiente manera
    $$A(t) = \Phi' (t) (\Phi(t))^{-1}$$
    $$A(t) = 
    \begin{pmatrix}
        0 & \cos(t)\\
        \cos(t) & 0\\
    \end{pmatrix} 
    \begin{pmatrix}
        1 & -\sin(t)\\
        -\sin(t) & 1\\
    \end{pmatrix}
    \frac{1}{\cos^2(t)} = 
    \begin{pmatrix}
        -\tan(t) & \frac{1}{\cos(t)}\\
        \frac{1}{\cos(t)} & -\tan(t)\\
    \end{pmatrix}
    $$

    Hemos encontrado una matriz $A(t)$ definida en un conjunto $I=\{t\in \R: \thinspace t\neq K\pi + \frac{\pi}{2}, \thinspace K\in\Z\}$, por lo tanto podemos afirmar 
    que en estas condiciones ambas soluciones $X_1$ y $X_2$ pueden ser solución de manera simultánea.\\ \\

    \textbf{Ejercicio 7. } \\


    

\end{document}
