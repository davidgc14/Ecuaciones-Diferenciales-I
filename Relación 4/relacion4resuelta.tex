\documentclass[fleqn]{article}

%\pgfplotsset{compat=1.17}

\usepackage{mathexam}
\usepackage{amsmath}
\usepackage{amsfonts}
\usepackage{graphicx}
\usepackage{systeme}
\usepackage{microtype}
\usepackage{multirow}
\usepackage{pgfplots}
\usepackage{listings}
\usepackage{tikz}
\usepackage{dsfont} %Numeros reales, naturales...

%\graphicspath{{images/}}
\newcommand*{\QED}{\hfill\ensuremath{\square}}

%Estructura de ecuaciones
\setlength{\textwidth}{15cm} \setlength{\oddsidemargin}{5mm}
\setlength{\textheight}{23cm} \setlength{\topmargin}{-1cm}



\author{David García Curbelo}
\title{Ecuaciones}

\pagestyle{empty}


\def\R{\mathds{R}}
\def\Z{\mathds{Z}}
\def\N{\mathds{N}}

\def\sup{$^2$}


\begin{document}

    \setcounter{page}{1}
    \pagestyle{plain}
    \markright{Relación 4 Ecuaciones Diferenciales}

    \begin{center}
        {\large\bf{Relación 4 Ecuaciones diferenciales}} \\
        \bf{Parte 1}\\
        
    \end{center}

    \textbf{Ejercicio 1.} \\

    Para comprobar que las funciones de la forma $ x(t) = A\cos(t) + B\sin(t)$ sean solución de la ecuación de orden dos $x'' + x = 0$, estudiemos
    las derivadas sucesivas de $x(t)$:
    $$x'(t)= - A\sin(t) + B\cos(t)$$
    $$ x''(t) = -A\cos(t) - B\sin(t)$$
    Sustituyendo en la ecuación obtenemos
    $$x''(t) + x(t) = 0 \Rightarrow -A\cos(t) - B\sin(t) + A\cos(t) + B\sin(t) = 0$$
    Como vemos que se cumple, podemos afirmar que las funciones de la forma $ x(t) = A\cos(t) + B\sin(t)$ son solución de la ecuación dada $x'' + x = 0$.

    Para el segundo apartado, en el que nos encontramos con la ecuación $x'''+ x' = 0$, la simplificaremos a una ecuación del tipo del apartado anterior
    y la resolveremos de la misma manera antes vista. Para ello vamos a considerar el siguiente cambio de variable
    $$y(t)=x'(t) \Rightarrow y''(t)=x'''(t) \Rightarrow y'' + y = 0$$ 
    La cual es una ecuación como la del apartado anterior. Allí vimos que $y(t)= A\cos(t) + B\sin(t)$ eran las soluciones no triviales para la nueva ecuación.
    Por tanto, deshaciendo el cambio obtenemos
    $$y(t)=x'(t)= A\cos(t) + B\sin(t) \Rightarrow x(t) = A\sin(t)-B\cos(t) + C, \quad C \in \R$$ \\ \\

    \textbf{Ejercicio 2.} \\
    
    Vamos a resolver este ejercicio mediante reducción al absurdo. Si $x(t)$ es solución del problema de valores iniciales planteado, entonces
    tiene que cumplirse $tx'-x=0$, y además $x(0)=1$ (con $t=0$), por lo que si sustituimos, la igualdad ha de mantenerse.
    $$tx'-x=tx'(t)-x(t)=0$$
    $$tx'(0)-x(0)=0 \Rightarrow t\cdot1 - 1 = 0 \Rightarrow t=1$$
    Lo cual es una contradicción puesto que hemos partido de la solución para $t_0=0$, por lo tanto hemos llegado a una contradicción y podemos
    concluir que este problema de valores iniciales carece de solución. \\ \\

    \textbf{Ejercicio 3.} \\

    Nos encontramos ante un problema de valores iniciales, donde la función y sus respectivas condiciones iniciales vienen dadas por 
    $$x''' + \cos(t)x'' + tx' +t^2x = 0$$
    $$x(1)=x'(1)=x''(1)=0$$
    Donde podemos ver que las funciones $ 1$ constante, $\cos(t)$, $t$ y $t^2$ son todas continuas y de clase $C^\infty(\R)$, 
    particularmente $C^1(\R)$. Entonces, por el Teorema de Existencia y unicidad para las ecuaciones lineales de orden superior, podemos 
    afirmar que existe una función de la forma $x(t) \in C^3(\R)$ que además es única. 
    Para proceder al cálculo de dicha solución, basta sustituir los valores iniciales dados en la función anterior. Tomamos $t_0=1$
    $$x'''(1) + \cos(1)x''(1) + 1x'(1) +1^2x(1) = x'''(1)=0$$
    $$x'''(1) = 0$$

    Como podemos ver, para $t=1$ obtenemos $$x(1)=x'(1)=x''(1)=x'''(1)=0$$
    y es fácil ver que una posible solución es la función $x(t)= 0$ constante, ya que cumple tanto las condiciones iniciales como la ecuación
    de orden superior dada. Y como sabemos por el teorema antes mencionado que la solución
    es única, podemos afirmar que tenemos la solución al problema de valores iniciales anteriormente planteado, dada por la función $x(t)= 0$ constante.\\ \\

    \textbf{Ejercicio 4.} \\

    Veamos si la función $x(t)= t^2e^t$ es solución de la ecuación lineal homogénea de orden 2. Para ello, supondremos que es solución y veremos si llegamos
    a una contradicción o no. Estudiemos sus derivadas sucesivas para poder sustituir en la ecuación
    $$x(t)= e^tt^2 \quad x'(t)= e^t(t^2 + 2t) \quad x''(t)= e^t(t^2 + 4t + 2)$$
    Sustituimos ahora en la ecuación $x'' + a_1x' + a_0x = 0$, con lo que obtenemos
    $$e^t(t^2 + 4t + 2) + e^ta_1(t^2 + 2t) + e^ta_0t^2 = 0 \Rightarrow (t^2 + 4t + 2) + a_1(t^2 + 2t) + a_0t^2 = 0 \Rightarrow a_0=-\frac{t^2 + 4t + 2 + a_1(t^2 + 2t)}{t^2}$$
    Con la que obtenemos una relación directa entre ambas funciones $a_0$ y $ a_1$, en la que podemos ver que el dominio en el que está definida no puede incluir
    ciertos puntos, por lo que no están definidas en $\R$. Tomando uno de estos puntos, por ejemplo el $t=0$, vemos que $x(0)=e^00^2=0$, por lo que dicha función 
    sí está definida en este punto, pero si sustituimos en la ecuación
    $$e^t(t^2 + 4t + 2) + e^ta_1(t^2 + 2t) + e^ta_0t^2 = 0, \quad con \quad t=0 \Rightarrow 2=0$$
    Por lo que concluimos que $x(t)= e^tt^2$ no es solución de la ecuación lineal homogénea de segundo orden.\\ \\

    \textbf{Ejercicio 5.} \\
    

    
    Partimos de que tanto $e^t$ como $e^-t$ son soluciones de nuestra ecuación de segundo orden $x'' + a_1x' + a_0x=0$. Si es así, entonces 
    \begin{itemize}
        \item $$x(t)=e^t \Rightarrow x''(t)=x'(t)=x(t)=e^t$$
                $$e^t + a_1e^t + a_0e^t = 0$$
        \item $$x(t)=e^{-t} \Rightarrow x''(t)=x(t)=e^t \quad x'(t)=-e^{-t}$$
                $$e^{-t} - a_1e^{-t} + a_0e^{-t} = 0$$
    \end{itemize}
    
    Con lo que obtenemos el siguiente sistema, que nos ayudará a calcular las funciones $a_0$ y $ a_1$ para poder estudiar el intervalo en el que 
    están definidas (y ver así si se trata de $I=\R$ o no):
    \begin{equation*}
        \left.
        \begin{aligned}
            e^t + a_1e^t + a_0e^t = 0 \\
            e^{-t} - a_1e^{-t} + a_0e^{-t} = 0
        \end{aligned}
        \right\} \Rightarrow a_0=-1, \quad a_1=0
    \end{equation*}
    Con estas funciones dadas de $a_0$ y $ a_1$ obtenemos la ecuación de segundo orden $x'' - x=0$, y por lo tanto $I=\R$.\\ \\

    \textbf{Ejercicio 6.} \\

    Como en el ejercicio anterior, partimos de que tanto $t$ como $et^2$ son soluciones de nuestra ecuación de segundo orden 
    $x'' + a_1x' + a_0x=0$. Si es así, entonces 
    \begin{itemize}
        \item $$x(t)=t \Rightarrow x''(t)=0, \quad x'(t)=1$$
                $$0 + a_1 + a_0t = 0$$
        \item $$x(t)=t^2 \Rightarrow x''(t)=2, \quad x'(t)=2t$$
                $$2 + a_12t + a_0t^2 = 0$$
    \end{itemize}

    Con lo que obtenemos el siguiente sistema, que nos ayudará a calcular las funciones $a_0$ y $ a_1$ para poder estudiar el intervalo en el que 
    están definidas (y ver así si se trata de $I=\R$ o no):
    \begin{equation*}
        \left.
        \begin{aligned}
            a_1 + a_0t = 0 \\
            2 + a_12t + a_0t^2 = 0
        \end{aligned}
        \right\} 
    \end{equation*}

    Una primera observación que se le puede hacer a este sistema es que $t$ no puede tomar el valor $0$, ya que si esto fuera así, el sistema no tendría
    solución (en el sistema, la segunda ecuación nos quedaría $2=0$). Por tanto ya podríamos afirmar que el intervalo $I\neq\R$, ya que el cero no se 
    encuentra en el dominio. Aun así calculemos las posibles soluciones de $a_0$ y $ a_1$, y tratemos de encontrar el intervalo $I$ en el que está 
    definida la ecuación. Por ello, considerando $t\neq 0$ obtenemos
    $$a_1,a_0: I \longrightarrow \R, \quad a_1=-\frac{2}{t}, \quad a_0=\frac{2}{t^2}$$
    Y concluimos que el intervalo $I$ puede ser cualquier abierto de la forma $I \subset \R - \{0\}$. \\ \\

    \textbf{Ejercicio 7.} \\

    Para el primer apartado, suponemos por hipótesis que existe un $t_0 \in \R$ tal que $W(f_1,f_2,f_3)(t_0) > 0$, donde ${f_1,f_2,f_3}$ sea un sistema fundamental
    de la ecuación dada $x''' + tx'' + a_1(t)x' + e^tx = 0$. Consideramos por tanto el Wronskiano  $W(f_1,f_2,f_3)(t)$ para un $t \geq t_0$, y usando la fórmula de 
    Lioville dada por
    $$W(f_1,f_2,f_3)(t) = W(f_1,f_2,f_3)(t_0) e^{-\int_{t_0}^t a_2(s)\thinspace ds}$$
    donde vemos en nuestra ecuación que $a_2(t)=t$, y que por tanto $-\int_{t_0}^t a_2(s)\thinspace ds = \frac{1}{2}(t_0^2 - t^2)$, y por tanto obtenemos
    $$W(f_1,f_2,f_3)(t_0) e^{\frac{1}{2}(t_0^2 - t^2)}, \quad t\geq t_0$$

    Para el segundo apartado estudiemos el límite dado por $\lim_{t\rightarrow\infty} W(f_1,f_2,f_3)(t)$. Para ello sustituimos el resultado obtenido en el apartado
    anterior, y el Wronskiano deja de tener la dependencia en $t$, para que el cálculo del límite sea más sencillo:
    $$\lim_{t\rightarrow\infty} W(f_1,f_2,f_3)(t) = \lim_{t\rightarrow\infty} W(f_1,f_2,f_3)(t_0)  e^{\frac{1}{2}(t_0^2 - t^2)} = $$
    $$\lim_{t\rightarrow\infty} W(f_1,f_2,f_3)(t_0) e^{\frac{t_0^2}{2}} e^{-\frac{t^2}{2}} = W(f_1,f_2,f_3)(t_0) e^{\frac{t_0^2}{2}} \cdot\lim_{t\rightarrow\infty}  e^{-\frac{t^2}{2}} =
    W(f_1,f_2,f_3)(t_0) e^{\frac{t_0^2}{2}} \cdot 0 = 0$$ \\ \\

    \textbf{Ejercicio 8.} \\

    Para este ejercicio, buscaremos una solución $f_2$ para la ecuación de segundo orden $x''-x=0$ usando la fórmula de Lioville usada también en el ejercicio anterior:
    $$W(f_1,f_2,f_3)(t) = W(f_1,f_2,f_3)(t_0) e^{-\int_{t_0}^t a_{k-1}(s)\thinspace ds}$$
    Siendo $k$ el orden de la ecuación (que en nuestro caso es $k=2$, y por tanto $a_1(t)=0$). Por definición del Wronskiano, construimos los respectivos determinantes y obtenemos
    $$
        \left|
        \begin{matrix}
            e^t & f_2(t) \\
            e^t & f_2'(t)
        \end{matrix}
        \right| = 
        \left|
        \begin{matrix}
            e^{t_0} & f_2(t_0) \\
            e^{t_0} & f_2'(t_0)
        \end{matrix}
        \right|
        \cdot e^{-\int_{t_0}^t 0\thinspace ds}
    $$   
    $$e^tf_2'(t) - e^tf_2(t) = e^{t_0} f_2'(t_0) - e^{t_0} f_2(t_0) = m$$
    Donde podemos ver que el segundo termino de la igualdad es una constante que denotaremos por $m$, ya que no tiene dependencia en $t$. Resolvemos por tanto la ecuación lineal 
    de primer orden dada por $e^tf_2'(t) - e^tf_2(t) = m$. Para ello utilizaremos técnicas vistas en el tema anterior para la resolución de ecuaciones diferenciales completas de primer orden.
    Por ello buscaremos primero una solución de la homogénea, y posteriormente buscaremos una solución particular para obtener todo el espacio de soluciones.
    \begin{itemize}
        \item Resolvemos la ecuación homogénea 
                $$e^tf_2'(t) - e^tf_2(t) = 0 \quad \Rightarrow \quad \frac{f_2'(t)}{f_2(t)}=1$$
                La cual vemos que se trata de una ecuación de variables separadas. Integrando en ambos lados obtenemos
                $$\int \frac{f_2'(t)}{f_2(t)} \thinspace dt = \int 1 \thinspace dt \quad \Rightarrow \quad f_2(t) = Ke^t$$
        \item Buscamos ahora una función $K(t)$ para que $f_2(t) = K(t)e^t$ sea solución de la ecuación completa. Para ello derivemos la ecuación obtenida en ambas partes de la igualdad
                $$f_2'(t) = K(t)e^t +K'(t)e^t$$
                Como sabemos que $f_2$ es solución de la ecuación $e^tf_2'(t) - e^tf_2(t) = m$, sustituimos y obtenemos
                $$m = e^t \left(K(t)e^t + K'(t)e^t - K(t)e^t \right) = K'(t)e^{2t} \quad \Rightarrow \quad K'(t)=e^{-2t}m \quad \Rightarrow \quad K(t)=-\frac{m}{2}e^{-2t} + C$$
                Y llegamos por tanto a la ecuación
                $$f_2(t) = -\frac{m}{2}e^{-2t}\cdot e^t + C\cdot e^t = -\frac{m}{2}e^{-t} + C e^t$$
                La cual vemos que es la solución general de todas las soluciones para este problema. Tomando ahora particularmente $m=-2$ y $C=0$ obtenemos la solución particular 
                $$f_2(t)=e^{-t}$$
                La cual es fácil ver que cumple la ecuación diferencial de segundo orden dada.\\ \\
    \end{itemize}

    \textbf{Ejercicio 9.} \\

    \framebox{$\Rightarrow$}
    Que $x_1$ y $x_2$ sean linealmente dependientes significa que existen $\lambda_1, \thinspace \lambda_2 \neq 0$ tal que $\lambda_1x_1 + \lambda_2x_2 = 0$,
    o lo que es lo mismo, existe un $\lambda \neq 0$ tal que $x_2=\lambda x_1$. Por lo tanto, el Wronskiano viene dado por 
    $$W(x_1,x_2)(t) = \left|
    \begin{matrix}
        x_1(t) & x_2(t) \\
        x_1'(t) & x_2'(t) \\
    \end{matrix}
    \right|= \left|
    \begin{matrix}
        x_1(t) & \lambda x_1(t) \\
        x_1'(t) & \lambda x_1'(t) \\
    \end{matrix}
    \right| = x_1'(t)\lambda x_1(t) - x_1'(t)\lambda x_1(t)=0, \quad \forall t \in I
    $$
Con lo que obtenemos que el Wronskiano es nulo cuando  $x_1$ y $x_2$ son linealmente dependientes.

\framebox{$\Leftarrow$} Razonamos de manera similar al apartado anterior. Partimos de que el Wronskiano es nulo:
$$W(x_1,x_2)(t) = \left|
\begin{matrix}
    x_1(t) & x_2(t) \\
    x_1'(t) & x_2'(t) \\
\end{matrix}
\right|=0 \quad \Rightarrow \quad x_1(t) x_2'(t) - x_1'(t)\lambda x_2(t)=0, \quad \forall t \in I$$
Con lo que obtenemos un tipo de ecuación diferencial que sabemos resolver. Suponiendo $x_1,x_2 \neq 0$ obtenemos lo que sigue
(es lógico suponer esto ya que si alguna de las dos fuera nula, ambas serían linealmente dependientes 
y por tanto habríamos acabado el ejercicio)
$$\frac{x_1'(t)}{x_1(t)}=\frac{x_2'(t)}{x_2(t)} \quad \Rightarrow \quad  \int \frac{x_1'(t)}{x_1(t)} dt=\int \frac{x_2'(t)}{x_2(t)} dt
\quad \Rightarrow \quad \ln x_1(t) = \ln x_2(t) +C, \quad C\in\R$$
Por lo tanto, mediante cálculos elementales llegamos a la igualdad $x_1(t)=k\cdot x_2(t)$ y por lo tanto deducimos que  $x_1$ y $x_2$ son linealmente dependientes.\\ \\

\textbf{Ejercicio 10.} \\

Existen muchas maneras de resolver este ejercicio, pero nosotros desarrollaremos la idea mas intuitiva y más lógica. Como queremos ver si $f_1(t) = e^t$ y $f_2(t)=\sin (t)$
pueden ser soluciones de la ecuación de segundo orden, supongamos por hipótesis que lo son. Si es así, sólo hace falta sustituir en nuestra ecuación 
las derivadas sucesivas de nuestras dos candidatas a solución (queremos que sean solución de la misma ecuación lineal de segundo orden).
\begin{equation*}
    \left.
    \begin{aligned}
        e^t + e^ta_1 + e^t a_0 = 0 \\
        -\sin + \cos(t)a_1 + \sin(t)a_0=0
    \end{aligned}
    \right\} \quad \Rightarrow \quad a_0(t)=\frac{\sin(t) + \cos(t)}{\sin(t) - \cos(t)}, \quad a_1(t)=\frac{2\sin(t)}{\sin(t) - \cos(t)} \thinspace \forall t \in I
\end{equation*}
Estos valores de $a_0$ y $a_1$ son sólo válidos cuando se cumpla la condición $\sin(t) \neq \cos(t)$, es decir, para todo $t \neq \frac{\pi}{4} + \pi K, \thinspace K \in \Z$.
Es decir, la ecuación lineal completa de segundo orden dada por 
$$x'' + x'\frac{2\sin(t)}{\sin(t) - \cos(t)} + x\frac{\sin(t) + \cos(t)}{\sin(t) - \cos(t)}$$
tendrá como solución $f_1(t) = e^t$ y $f_2(t)=\sin (t)$ simultáneamente cuando esté definida en intervalos abiertos y conexos de la forma 
$$I = \left(\frac{\pi}{4} + \pi K, \frac{\pi}{4} + \pi (K+1)\right), \quad K \in \Z$$
Por lo tanto podemos afirmar que las funciones $f_1(t) = e^t$ y $f_2(t)=\sin (t)$ pueden ser solución de la ecuación lineal completa siempre que dicha ecuación
esté definida en un intervalo como el que acabamos de ver.\\ \\

\textbf{Ejercicio 11.} \\





\end{document}