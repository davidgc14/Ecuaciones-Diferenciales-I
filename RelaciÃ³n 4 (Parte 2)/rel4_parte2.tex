\documentclass[fleqn]{article}

%\pgfplotsset{compat=1.17}

\usepackage{mathexam}
\usepackage{amsmath}
\usepackage{amsfonts}
\usepackage{graphicx}
\usepackage{systeme}
\usepackage{microtype}
\usepackage{multirow}
\usepackage{pgfplots}
\usepackage{listings}
\usepackage{tikz}
\usepackage{dsfont} %Numeros reales, naturales...
\usepackage{cancel}

%\graphicspath{{images/}}
\newcommand*{\QED}{\hfill\ensuremath{\square}}

%Estructura de ecuaciones
\setlength{\textwidth}{15cm} \setlength{\oddsidemargin}{5mm}
\setlength{\textheight}{23cm} \setlength{\topmargin}{-1cm}



\author{David García Curbelo}
\title{Ecuaciones}

\pagestyle{empty}


\def\R{\mathds{R}}
\def\Z{\mathds{Z}}
\def\N{\mathds{N}}

\def\sup{$^2$}

\def\next{\quad \Rightarrow \quad}

\begin{document}

    \setcounter{page}{1}
    \pagestyle{plain}
    \markright{Relación 4 Ecuaciones Diferenciales}

    \begin{center}
        {\large\bf{Relación 4 Ecuaciones diferenciales}} \\
        \bf{Parte 2}\\
        
    \end{center}

    \textbf{Ejercicio 1. } \\

    Nos piden encontrar la solución al sistema dado por 
    \begin{equation*}
        \left.
        \begin{aligned}
            x' + ty &= -1\\
            y' + x' &= -1\\
        \end{aligned}
        \right\}\quad \Rightarrow \quad
        \left.
        \begin{aligned}
            x' &= -ty -1\\
            y' &= ty + 1 - 1\\
        \end{aligned}
        \right\}\quad \Rightarrow \quad
        \left.
        \begin{aligned}
            x' &= -ty -1\\
            y' &= ty\\
        \end{aligned}
        \right\} 
    \end{equation*}
    Dicho sistema podemos expresar de forma $X'(t)=A(t)X(t) + B(t)$, donde vemos que cada uno de los elementos viene dado por

    \begin{equation*}
        X'(t) = 
        \begin{pmatrix}
            x' \\
            y' \\
        \end{pmatrix}\\
        A(t) = 
        \begin{pmatrix}
            0 & -t  \\
            0 & t  \\
        \end{pmatrix}\\
        X(t) = 
        \begin{pmatrix}
            x \\
            y \\
        \end{pmatrix}\\
        B(t) = 
        \begin{pmatrix}
            -1 \\
            0 \\
        \end{pmatrix}\\
    \end{equation*}
    Nos encontramos ante un sistema de ecuaciones diferenciales lineales de primer orden. Podemos ver que la segunda ecuación sólo tiene coeficientes
    en $y$,  ya que viene dada por $y' = ty$, la cual sabemos que tiene por solución
    $$y(t) = K\cdot e^{\frac{t^2}{2}}, \quad K\in \R$$
    Sustituyendo en la primera ecuación, la cual viene dada por $x' = -ty -1$, que sustituyendo la solución obtenida de la ecuación anterior obtenemos
    $$x'(t) = -tKe^{\frac{t^2}{2}} - 1$$
    Que integrando en ambos lados resulta
    $$x(t) = - Ke^{\frac{t^2}{2}} - t + C, \quad C\in \R$$
    y obtenemos así la solución general del sistema planteado, en función de un parametro $K\in\R$, el cual variará em función de la condición inicial.
    $$X(t) = \begin{pmatrix}
        Ke^{\frac{t^2}{2}} - t + C\\
        Ke^{\frac{t^2}{2}} \\
    \end{pmatrix}, \quad K\in \R$$
    
    Y ahora, para el cálculo de una matriz solución particular, bastará con sustituir los parámetros que tenemos para obtener dos soluciones distintas del problema
    inicial. Tomemos por ello
    \begin{equation*}
        x_1(t) = 
        \begin{pmatrix}
            -t \\
            0 \\
        \end{pmatrix}\\
        x_2(t) = 
        \begin{pmatrix}
            e^{\frac{t^2}{2}} - t \\
            e^{\frac{t^2}{2}} \\
        \end{pmatrix}
    \end{equation*}
    Donde hemos tomado para la primera solución $K=C=0$, y para la segunda $K=1, \thinspace C=0 $, y por ello podemos construir su matriz solución dada por:
    $$
        \Phi (t) = 
        \begin{pmatrix}
            -t & -e^{\frac{t^2}{2}} - t \\
            0 & e^{\frac{t^2}{2}} \\           
        \end{pmatrix}
    $$\\ \\

    \newpage 

    \textbf{Ejercicio 2. } \\

    Como tenemos la matriz $\Phi$, supongamos que es matriz solución de un sistema dado $X'(t) = A(t) X(t)$. Por ser matriz solución (y ser clase $\mathcal{C}^1$ y por tanto
    ser derivable), sabemos que tiene que cumplirse $\Phi' = A\Phi$. Como el determinante de la matriz $\Phi$ es no nulo $\forall t \in I\subset \R$, podemos afirmar que dicha matriz 
    tiene inversa, y que por lo tanto la matriz $A$ puede representarse de la siguiente manera
    $$A=\Phi' \Phi^{-1}$$
    Por ello vemos que hemos determinado la matriz $A$ en función de $\Phi$, para la cual, el problema $X'(t) = A(t) X(t)$ tiene por matriz solución la matriz $\Phi$, y que  por lo tanto (por tener
    determinante no nulo) podemos afirmar que se trata de una matriz fundamental.\\ \\

    \textbf{Ejercicio 3. } \\

    Supondremos durante la demostración que ambas matrices son clase $\mathcal{C}^1$. Si queremos probar la existencia de un $C \in \R$ tal que se cumpla $\Phi = \Psi \cdot C$, será lo mismo probar que $C = \Phi \Psi^{-1}$ 
    (podemos afirmar que tiene inversa ya que tiene determinante no nulo) y por tanto 
    $$(\Psi^{-1}\Phi)' = 0$$
    Por tanto veamos si, siendo ambas matrices fundamentales, estudiemos el valor de la derivada. 
    $$(\Psi^{-1}\Phi)' = \Psi^{-1} \Phi'  + (\Psi^{-1})' \Phi $$
    Estudiemos a parte el valor de $(\Psi^{-1})'$. Para ello consideremos $\Psi \Psi^{-1} = 1_n$, y por tanto la derivada ha de ser nula, 
    $$(\Psi^{-1}\Psi)' = \Psi' \Psi^{-1} + \Psi (\Psi^{-1})'= 0 \next (\Psi^{-1})' = - \Psi^{-1}\Psi' \Psi^{-1}$$
    Podemos considerar inversas puesto que, al tratarse de matrices fundamentales tienen determinante no nulo. Retomemos por tanto la ecuación anterior, 
    $$(\Psi^{-1}\Phi)' = \Psi^{-1}\Phi' + (\Psi^{-1})' \Phi  = \Psi^{-1} \Phi' - \Psi^{-1}\Psi' \Psi^{-1} \Phi  $$
    Como ambas matrices son fundamentales, son por ello solución de $X'(t) = A(t) X(t)$ y por tanto la cumplen $\Phi' = A\Phi$ y $\Psi' = A\Psi$. Sustituimos y obtenemos
    $$(\Psi^{-1}\Phi)' = \Psi^{-1} A(t)\Phi - \Psi^{-1}A(t) \cancel{\Psi \Psi^{-1}} \Phi = \Psi^{-1} A(t)\Phi - \Psi^{-1} A(t)\Phi = 0$$
    $$(\Psi^{-1}\Phi)' = 0$$
    Como queríamos probar.\\ \\

    \textbf{Ejercicio 4. } \\

    Repetimos la demostración hecha en el apartado anterior. Como nuestra matriz es clase $\mathcal{C}^1$ con determinante no nulo, podemos afirmar que tiene inversa. También podemos afirmar
    la obviedad $\Phi \Phi^{-1} = 1_n$, y por tanto si derivamos ambos lados de la igualdad obtenemos
    $$(\Phi^{-1}\Phi)' = \Phi' \Phi^{-1} + \Phi (\Phi^{-1})'= 0 \next - \Phi' \Phi^{-1} = \Phi (\Phi^{-1})'$$
    Como hemos dicho que tiene inversa, basta despejar $(\Phi^{-1})'$ y obtenemos el resultado que buscábamos
    $$(\Phi^{-1})' = - \Phi^{-1} \Phi' \Phi^{-1} $$\\ \\

    \textbf{Ejercicio 5. } \\
    El enunciado es falso. Consideremos un sistema homogéneo $X'(t) = A X(t)$ (con la matriz $A$ con coeficientes constantes) tal que $\Phi$ sea una matriz fundamental (clase $\mathcal{C}^1$) del sistema homogeneo
    dado. Por tanto sabemos que $\Phi$ es solución del sistema, por lo que obtenemos 
    $$\Phi' = A \Phi$$
    Como hemos supuesto que la matriz $A$ tiene coeficientes constantes, al derivar en ambos lados de la igualdad obtenemos 
    $$\Phi'' = A \Phi'$$
    Luego $\Phi'$ también es solución del sistema homogéneo $X'(t) = A X(t)$.
    
    
    
    Toda matriz fundamental tiene determinante no nulo, por lo que basta tomar una matriz $\Phi$ tal que $det(\Phi) = C, \thinspace C \in \R-\{0\}$,
    con lo que tenemos una matriz con determinante no nulo para todo $t$ en su dominio de definición\\ \\

    \textbf{Ejercicio 6. } \\

    Para ver si son solución de manera simultánea, podemos considerar la matriz solución 
    $$\Phi (t) = \begin{pmatrix}
        1 & \sin(t)\\
        \sin(t) & 1\\
    \end{pmatrix}$$
    Y comprobar si existe una matriz $A(t)$ para que la matriz $\Phi$ sea solución del sistema homogéneo $X'(t) = A(t) X(t)$. Si suponemos que sea matriz solución,
    entonces tiene que cumplirse
    $$\Phi' (t) = A(t) \Phi(t)$$
    $$
    \begin{pmatrix}
        0 & \cos(t)\\
        \cos(t) & 0\\
    \end{pmatrix} 
    = A(t)
    \begin{pmatrix}
        1 & \sin(t)\\
        \sin(t) & 1\\
    \end{pmatrix}
    $$
    Si estudiamos el determinante de la matriz $\Phi$ vemos que $det(\Phi)=1-\sin^2(t)=\cos^2(t)$, el cual vemos que es no nulo para todo 
    $t\neq K\pi + \frac{\pi}{2}, \thinspace K\in\Z$. Por lo tanto, considerando dicho determinante no nulo, podemos afirmar que tiene inversa, y por tanto 
    podemos despejar la matriz $A(t)$ de la siguiente manera
    $$A(t) = \Phi' (t) (\Phi(t))^{-1}$$
    $$A(t) = 
    \begin{pmatrix}
        0 & \cos(t)\\
        \cos(t) & 0\\
    \end{pmatrix} 
    \begin{pmatrix}
        1 & -\sin(t)\\
        -\sin(t) & 1\\
    \end{pmatrix}
    \frac{1}{\cos^2(t)} = 
    \begin{pmatrix}
        -\tan(t) & \frac{1}{\cos(t)}\\
        \frac{1}{\cos(t)} & -\tan(t)\\
    \end{pmatrix}
    $$

    Hemos encontrado una matriz $A(t)$ definida en un conjunto $I=\{t\in \R: \thinspace t\neq K\pi + \frac{\pi}{2}, \thinspace K\in\Z\}$, por lo tanto podemos afirmar 
    que en estas condiciones ambas soluciones $X_1$ y $X_2$ pueden ser solución de manera simultánea.\\ \\

    \textbf{Ejercicio 7. } \\

    Si dicha solución existiera, tendría que cumplir el sistema completo dado $X'(t) = AX(t) + e^{\alpha t}V$. Por ello, estudiemos las derivadas de nuestra candidata a solución
    $$X_1(t) = e^{\alpha t}W \quad X_1'(t) = \alpha e^{\alpha t}W$$
    Sustituyendo en el sistema antes mencionado obtenemos
    $$X_1'(t) = AX_1(t) + e^{\alpha t}V  \next  \alpha e^{\alpha t}W = A \cdot e^{\alpha t}W + e^{\alpha t}V  \next  \alpha W = A\cdot W + V  \next V = (\alpha I_n -A) W$$
    Para poder despejar el vector $W$ tenemos que ver primero si $(\alpha I_n -A)$ tiene inversa, es decir, si su determinante es nulo o no. Si fuera nulo, por la fórmula del polinomio 
    característico, podríamos afirmar que $\alpha$ es valor propio de $A$, lo cual no es cierto, por lo que concluimos que $det(\alpha I_n -A)\neq 0$ y por tanto tiene inversa. 
    $$ W = (\alpha I_n -A)^{-1} V $$
    Y hemos encontrado un vector para que $X_1$ sea solución, y por lo tanto podemos afirmar que existe un cierto $W$ tal que $X_1(t) = e^{\alpha t}W$ es solución del sistema completo
    $X'(t) = AX(t) + e^{\alpha t}V$.\\ \\

    \textbf{Ejercicio 8. } \\

    Consideremos la forma matricial del sistema dada por
    $$
    X'(t) = AX(t) + B(t) = 
    \begin{pmatrix}
        2 & -3 \\
        1 & -2 \\
    \end{pmatrix}
    X(t) + 
    \begin{pmatrix}
        3e^{2t} \\
        -8e^{-3t}
    \end{pmatrix}
    =
    \begin{pmatrix}
        2 & -3 \\
        1 & -2 \\
    \end{pmatrix}
    X(t) + e^{-3t}
    \begin{pmatrix}
        3e^{5t} \\
        -8
    \end{pmatrix}
    $$
    Podemos ver fácilmente que los valores propios de la matriz $A$ son $\lambda_1=1$ y $\lambda_2=-1$, ambos distintos de $\alpha = -3$, por lo que aplicando el ejercicio anterior sabemos 
    que una solución particular del sistema viene dada por 
    $$X_p(t) = e^{\alpha t}W =  e^{-3t}W  = e^{-3t}(\alpha I_n -A)^{-1} V = e^{-3t}(\alpha I_n -A)^{-1} \begin{pmatrix} 3e^{5t} \\ -8 \end{pmatrix}$$
    Calculemos por ello la matriz $(\alpha I_n -A)^{-1}$ para proceder después al cálculo de la solución particular (sabemos que existe inversa porque $\alpha$ no es valor propio)
    $$
    (\alpha I_n -A)^{-1} = 
    \begin{pmatrix}
        -5 & 0\\
        -4 & -1
    \end{pmatrix}^{-1}=
    \begin{pmatrix}
        -1/5 & 0\\
        4/5 & -1
    \end{pmatrix}
    $$
    $$
    \Rightarrow X_p(t) = 
    e^{-3t}(\alpha I_n -A)^{-1} 
    \begin{pmatrix} 
    3e^{5t} \\ 
    -8 
    \end{pmatrix} = 
    e^{-3t}
    \begin{pmatrix}
        -1/5 & 0\\
        4/5 & -1   
    \end{pmatrix}
    \begin{pmatrix} 
        3e^{5t} \\ 
        -8 
    \end{pmatrix} = 
    e^{-3t}
    \begin{pmatrix}
        -e^{5t}3/5 \\
        e^{5t}12/5+8
    \end{pmatrix}
    $$
    Con lo que hemos obtenido la solución particular. Calculemos ahora el sistema fundamental. Como ya teníamos los valores propios $\lambda_1=1$ y $\lambda_2=-1$, sólo nos falta conocer
    sus vectores propios asociados, los cuales vienen dados por $V_1 = \begin{pmatrix} 3\\1 \end{pmatrix}$ y $V_2 = \begin{pmatrix} 1\\1 \end{pmatrix}$, con lo que ya tenemos todos los elementos
    para construir nuestra solución general
    $$X(t) = c_1e^{\lambda_1 t}V_1 + c_2e^{\lambda_2 t}V_2 + c_3X_p(t) = c_1e^{t}\begin{pmatrix} 3\\1 \end{pmatrix} + c_2e^{-t} \begin{pmatrix} 1\\1 \end{pmatrix} + 
    e^{-3t}\begin{pmatrix}-e^{5t}\frac{3}{5} \\e^{5t}\frac{12}{5}+8\end{pmatrix}, \quad c_1,c_2 \in \R$$\\ \\

    \textbf{Ejercicio 9. } \\
    
    Para encontrar la solución general, encontremos primero una solución del sistema homogéneo asociado y le añadiremos después una solución particular. Para encontrar
    un sistema fundamental del sistema homogéneo asociado consideremos los valores propios y sus vectores propios asociados de la matriz $A$ para encontrar soluciones de la forma $e^{\alpha t}V$.
    Vemos que la matriz $A$ tiene por polinomio característico $p_A(\lambda) = \lambda^2 - 3\lambda + 2$ cuyas raices son $\lambda_1=1$ y $\lambda_2=2$, y sus vectores propios asociados vienen dados por
    $V_1 = \begin{pmatrix} 1\\1 \end{pmatrix}$ y $V_2 = \begin{pmatrix} 2\\1 \end{pmatrix}$ respectivamente. por ello tenemos las soluciones de la homogénea
    \begin{equation*}
        X_1(t)=e^t \begin{pmatrix} 1\\1 \end{pmatrix}\\
        X_2(t)=e^{2t} \begin{pmatrix} 2\\1 \end{pmatrix}\\
    \end{equation*}
    Las cuales vemos que son linealmente independientes ya que su matriz solución asociada tiene determinante no nulo $ \forall t \in \R$, y por tanto es una matriz fundamental. Calculemos ahora una solución particular del sistema completo.
    Para ello usaremos el método de variación de constantes, mediante el cual sabemos que la solución particular viene dada por 
    $$X_p(t) = \Phi(t) \int \Phi^{-1}(t)B(t) \thinspace dt$$
    donde la matriz $\Phi$ es la matriz fundamental, la cual sabemos que tiene inversa ya que su determinante no se anula. 
    \begin{equation*}
        \Phi(t) = 
        \begin{pmatrix}
            e^t & 2e^{2t}\\
            e^t & e^{2t}
        \end{pmatrix}\\
        \Phi^{-1}(t) = \frac{1}{e^{2t}}
        \begin{pmatrix}
            -e^t & 2e^t\\
            1 & -1
        \end{pmatrix}
    \end{equation*}

    $$X_p(t) = \Phi(t) \int 
    \frac{1}{e^{2t}}
    \begin{pmatrix}
        -e^t & 2e^t\\
        1 & -1
    \end{pmatrix}
    \begin{pmatrix}
        e^t \\ 2
    \end{pmatrix}
    \thinspace dt = \Phi(t)
    \begin{pmatrix}
        -t - 4e^{-t}\\
        e^{-2t} - e^{-t}
    \end{pmatrix}
    +C = 
    \begin{pmatrix}
        -e^t(t+2) -2\\
        -e^t(t+1) -3
    \end{pmatrix}
    +C, \quad C \in \R^2
    $$
    Como se trata de una solución particular, podemos considerar $C=\begin{pmatrix} 2 \\ 3\end{pmatrix}$, con lo que obtenemos la solución particular
    $$X_p(t)=
    \begin{pmatrix}
        -e^t(t+2) -2\\
        -e^t(t+1) -3
    \end{pmatrix}$$
    Y por tanto la solución general viene dada por 

    $$X(t)= 
    \begin{pmatrix}
        e^t & 2e^{2t}\\
        e^t & e^{2t}
    \end{pmatrix}
    C_1 + 
    \begin{pmatrix}
        -e^t(t+2) -2\\
        -e^t(t+1) -3
    \end{pmatrix}, \quad C_1\in\R^2
    $$\\ \\

    \textbf{Ejercicio 10. } \\

    Calculemos la solución como en el apartado anterior, calculando la solución de la homogénea y posteriormente una solución particular.Podemos ver que los valores propios 
    de $A$ son $\lambda_1=2(1+i)$ y $\lambda_2=2(1-i)$, con vectores propios asociados $V_1 = \begin{pmatrix} -i \\ 2 \end{pmatrix}$ y $V_2 = \begin{pmatrix} i \\ 2 \end{pmatrix}$
    respectivamente. Para estudiar las soluciones de la homogénea, tomemos un valor propio, por ejemplo $\lambda_1$ y calculemos su solución asociada
    $$\hat{X}(t) = V_1e^{t\lambda_1} = \begin{pmatrix} -i \\ 2 \end{pmatrix} e^{2t} e^{2ti} = \begin{pmatrix} -i \\ 2 \end{pmatrix} e^{2t} (\cos(2t) + i\sin(2t))=
    e^{2t} \begin{pmatrix} 
        -i\cos(2t) + \sin(2t)\\
        2(\cos(2t) + i\sin(2t))
    \end{pmatrix}
    $$
    $$\Rightarrow \hat{X}(t) = e^{2t}\left[
    \begin{pmatrix}
        \sin(2t)\\
        2\cos(2t)
    \end{pmatrix}
    + i
    \begin{pmatrix}
        -\cos(2t)\\
        2\sin(2t)
    \end{pmatrix}
    \right]
    $$
    Y podemos considerar por ello la solución general dada por 
    $$
    X(t) = c_1 e^{2t}
    \begin{pmatrix}
        \sin(2t)\\
        2\cos(2t)
    \end{pmatrix}
    + c_2 e^{2t}
    \begin{pmatrix}
        -\cos(2t)\\
        2\sin(2t)
    \end{pmatrix}
    +X_p(t), \quad c_1,c_2 \in \R
    $$
    Calculemos ahora la solución particular para obtener la solución general del sistema completo. Para ello podemos considerar el resultado obtenido en el ejercicio 7 anteriormente calculado,
    ya que podemos representar $B(t)=\begin{pmatrix} te^{2t} \\ -e^{2t} \end{pmatrix} = e^{2t}\begin{pmatrix} t \\ -1 \end{pmatrix}$, y como sabemos que ninguno de los dos valores propios vale $2$,
    podemos calcular la solución particular mediante
    $$X_p(t) = e^{\alpha t} (\alpha I_2 -A)^{-1} V = e^{2t} (2I_2 - A)^{-1} \begin{pmatrix} t \\ -1 \end{pmatrix} = 
    e^{2t} 
    \begin{pmatrix}
        0 & 1/6 \\
        1 & 0
    \end{pmatrix}
    \begin{pmatrix} t \\ -1 \end{pmatrix} = 
    e^{2t} \begin{pmatrix} -\frac{1}{6} \\ t \end{pmatrix} 
    $$
    Y por tanto hemos conseguido la solución general del problema
    $$
    X(t) = c_1 e^{2t}
    \begin{pmatrix}
        \sin(2t)\\
        2\cos(2t)
    \end{pmatrix}
    + c_2 e^{2t}
    \begin{pmatrix}
        -\cos(2t)\\
        2\sin(2t)
    \end{pmatrix}
    + e^{2t} \begin{pmatrix} -\frac{1}{6} \\ t \end{pmatrix}, \quad c_1,c_2 \in \R
    $$\\ \\

    \textbf{Ejercicio 11. } \\
    
    Nos encontramos ante un problema de valores iniciales, para el que necesitaremos primero calcular la solución general del sistema completo para luego aplicarle la condición inicial.
    Calculemos primero una matriz solución para el sistema homogéneo asociado, para el cual sabemos que las soluciones vienen dadas por $X_k(t)=e^{t\lambda_k}V_k$, donde $\lambda_k$ es un 
    valor propio de la matriz $A$ y $V_k$ su vector propio asociado. Es fácil calcular sus valores propios $\lambda_1 = 3+5i$ y $\lambda_1 = 3-5i$, con vectores propios asociados
    $V_1 = \begin{pmatrix} -i\\1 \end{pmatrix}$ y $V_2 = \begin{pmatrix} i\\1 \end{pmatrix}$ respectivamente. Tomemos un valor propio, por ejemplo $\lambda_1$ y calculemos su solución asociada

    $$\hat{X}(t) = V_1e^{t\lambda_1} = \begin{pmatrix} -i \\ 1 \end{pmatrix} e^{3t} e^{5ti} = \begin{pmatrix} -i \\ 1 \end{pmatrix} e^{3t} (\cos(5t) + i\sin(5t))=
    e^{3t}
    \begin{pmatrix}
        -i\cos(5t) + \sin(5t)\\
        \cos(5t) + i\sin(5t)   
    \end{pmatrix}    
    $$
    $$
    \Rightarrow \hat{X}(t) = e^{3t} \left[
    \begin{pmatrix}
        \sin(5t) \\
        \cos(5t)
    \end{pmatrix}
    +i
    \begin{pmatrix}
        -\cos(5t)\\
        \sin(5t)
    \end{pmatrix}
    \right]
    $$
    Con lo que podemos considerar la solución general dada por 
    $$
    X(t) = 
    c_1e^{3t}
    \begin{pmatrix}
        \sin(5t) \\
        \cos(5t)
    \end{pmatrix}
    +c_2e^{3t}
    \begin{pmatrix}
        -\cos(5t)\\
        \sin(5t)
    \end{pmatrix}
    +X_p(t), \quad c_1,c_2 \in \R
    $$
    Calculemos ahora una solución particular del sistema completo. Para ello consideremos primero la matriz fundamental dada por 
    $$\Phi(t)=
    \begin{pmatrix}
        e^{3t}\sin(5t) & -e^{3t}\cos(5t)
        e^{3t}\cos(5t) & e^{3t}\sin(5t)
    \end{pmatrix}
    $$
    Cuyo determinante vemos que es no nulo para todo $t\in\R$, y por tanto sabemos que tiene inversa. Procedemos al cálculo de la particular, la cual viene dada por la ecuación
    $$X_p(t) = \Phi(t) \int \Phi^{-1}(t)B(t) \thinspace dt$$
    $$X_p(t) = \Phi(t) \int 
    \begin{pmatrix}
        e^{-3t}\sin(5t) & e^{-3t}\cos(5t)
        -e^{-3t}\cos(5t) & e^{-3t}\sin(5t)
    \end{pmatrix}
    \begin{pmatrix}
        e^{-t} \\ 0
    \end{pmatrix}
    dt = \Phi(t)
    \begin{pmatrix}
        e^{-4t}\sin(5t) \\ e^{-4t}\cos(5t)
    \end{pmatrix}
    =
    \begin{pmatrix}
        \frac{1}{e^{t}} \\ 0
    \end{pmatrix}
    $$
    Con lo que obtenemos la solución general
    $$
    X(t) = 
    c_1e^{3t}
    \begin{pmatrix}
        \sin(5t) \\
        \cos(5t)
    \end{pmatrix}
    +c_2e^{3t}
    \begin{pmatrix}
        -\cos(5t)\\
        \sin(5t)
    \end{pmatrix}
    +
    \begin{pmatrix}
        \frac{1}{e^{t}} \\ 0
    \end{pmatrix}
    , \quad c_1,c_2 \in \R
    $$
    Imponemos ahora la condición inicial para poder obtener los valores de $c_1$ y $c_2$ y obtener así la solución al problema de valores iniciados planteado
    $$X(0)=
    c_1
    \begin{pmatrix}
        0 \\ 1
    \end{pmatrix}
    + c_2
    \begin{pmatrix}
        -1 \\ 0
    \end{pmatrix}
    +
    \begin{pmatrix}
        1 \\ 0
    \end{pmatrix}
    =
    \begin{pmatrix}
        0 \\ 1
    \end{pmatrix}
    \next 
    \begin{pmatrix}
        -c_2 +1 \\ c_1
    \end{pmatrix}
    =
    \begin{pmatrix}
        0 \\ 1
    \end{pmatrix}
    \next c_1=c_2=1
    $$
    Y por lo tanto la solución general al problema de valores iniciales planteado es
    $$
    X(t) = 
    e^{3t}
    \begin{pmatrix}
        \sin(5t) \\
        \cos(5t)
    \end{pmatrix}
    +e^{3t}
    \begin{pmatrix}
        -\cos(5t)\\
        \sin(5t)
    \end{pmatrix}
    +
    \begin{pmatrix}
        \frac{1}{e^{t}} \\ 0
    \end{pmatrix}
    $$\\ \\

    \textbf{Ejercicio 12. } \\

    Para transcribir la ecuación en forma de sistema de ecuaciones lineal basta con considerar los vectores $X(t)$ y $B(t)$ y la matriz $A(t)$ dados de la siguiente forma
    $$
    X(t) = 
    \begin{pmatrix}
        x(t) \\
        x'(t) \\
        x''(t) \\
    \end{pmatrix}\\
    \quad \quad
    B(t)=
    \begin{pmatrix}
        0 \\ 0 \\ b(t)
    \end{pmatrix}
    =
    \begin{pmatrix}
        0 \\ 0 \\ e^t
    \end{pmatrix}
    $$
    $$
    A(t)=
    \begin{pmatrix}
        0 & 1 & 0\\
        0 & 0 & 1\\
        -a_0(t) & -a_1(t) & -a_2(t)\\
    \end{pmatrix}
    =
    \begin{pmatrix}
        0 & 1 & 0\\
        0 & 0 & 1\\
        6 & -11 & 6\\
    \end{pmatrix}\\
    $$
    Con lo que obtenemos así el sistema matricial
    $$
    X'(t)=
    \begin{pmatrix}
        0 & 1 & 0\\
        0 & 0 & 1\\
        6 & -11 & 6\\
    \end{pmatrix}\\
    X(t) + 
    \begin{pmatrix}
        0 \\ 0 \\ e^t
    \end{pmatrix}
    $$

    Para encontrar la solución del sistema empleamos el mismo método visto en los ejercicios anteriores. Encontremos un sistema fundamental del sistema homogéneo asociado donde cada solución tiene la forma que ya 
    sabemos $X_k(t) = e^{\lambda_k t}V_k$ donde $\lambda_k$ y $V_k$ son los valores propios y vectores propios asociados respectivamente. Estudiamos por ello los valores propios de la matriz $A(t)$,
    donde vemos fácilmente que vienen dados por $\lambda_1 = 1$, $\lambda_2 = 2$ y $\lambda_3 = 3$, con vectores propios asociados 
    $V_1=\begin{pmatrix} 1 \\ 1 \\ 1 \end{pmatrix}$, $V_2=\begin{pmatrix} 1 \\ 2 \\ 4 \end{pmatrix}$ y $V_3=\begin{pmatrix} 1 \\ 3 \\ 9 \end{pmatrix}$, respectivamente. por ello hemos obtenido la matriz solución
    $$
    \Phi(t)=
    \begin{pmatrix}
        e^t & e^{2t} & e^{3t} \\
        e^t & 2e^{2t} & 3e^{3t} \\ 
        e^t & 4e^{2t} & 9e^{3t}
    \end{pmatrix}
    $$
    Que como podemos ver, su determinante $det(\Phi(t))=2e^{6t}$ el cual es distinto de cero para todo $t\in\R$, por lo que podemos afirmar de que se trata de una matriz fundamental
    (y que por tanto tiene inversa). Encontremos ahora una solución particual del sistema completo. Es fácil calcular $\Phi^{-1}$, por lo que podemos aplicar la igualdad siguiente para
    encontrar una solución particular
    $$X_p(t) = \Phi(t) \int \Phi^{-1}(t)B(t) \thinspace dt$$
    $$
    X_p(t) = \Phi(t) \int e^{3t}
    \begin{pmatrix}
        3e^{2t} & -\frac{5e^{2t}}{2} & \frac{e^{2t}}{2} \\
        -3e^t & 4e^t & -e^t \\
        1 & -\frac{3}{2} & \frac{1}{2}
    \end{pmatrix}
    \begin{pmatrix}
        0 \\ 0 \\ e^t
    \end{pmatrix}
    = \Phi(t) \int
    \begin{pmatrix}
        \frac{1}{2} \\ -\frac{1}{e^t} \\ \frac{1}{2e^{2t}}
    \end{pmatrix}
    = 
    \begin{pmatrix}
        e^t & e^{2t} & e^{3t} \\
        e^t & 2e^{2t} & 3e^{3t} \\ 
        e^t & 4e^{2t} & 9e^{3t}
    \end{pmatrix}\left[
    \begin{pmatrix}
        \frac{t}{2} \\ \frac{1}{e^t} \\ -\frac{1}{4e^{2t}}
    \end{pmatrix}
    +C
    \right]
    $$
    Con $C\in\R$. Como buscamos una solución particular, podemos considerar la constante $C$ nula, y obtenemos así la solución particular dada por 
    $$X_p(t)=\frac{1}{4}
    \begin{pmatrix}
        2e^{t}t + 3e^t \\
        2e^{t}t + 5e^t \\
        2e^{t}t + 7e^t \\
    \end{pmatrix}
    $$
    Y obtenemos así la solución general
    $$X(t) = 
    \begin{pmatrix}
        e^t & e^{2t} & e^{3t} \\
        e^t & 2e^{2t} & 3e^{3t} \\ 
        e^t & 4e^{2t} & 9e^{3t}
    \end{pmatrix}C
    + \frac{1}{4}
    \begin{pmatrix}
        2e^{t}t + 3e^t \\
        2e^{t}t + 5e^t \\
        2e^{t}t + 7e^t \\
    \end{pmatrix}
    , \quad C\in\R^3
    $$
    Y por tanto tenemos la solución al problema planteado
    $$x(t) = e^tc_1 + e^{2t}c_2 + e^{3t}c_3 + \frac{1}{4}(2e^{t}t + 3e^t). $$

\end{document}
