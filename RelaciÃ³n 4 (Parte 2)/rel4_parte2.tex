\documentclass[fleqn]{article}

%\pgfplotsset{compat=1.17}

\usepackage{mathexam}
\usepackage{amsmath}
\usepackage{amsfonts}
\usepackage{graphicx}
\usepackage{systeme}
\usepackage{microtype}
\usepackage{multirow}
\usepackage{pgfplots}
\usepackage{listings}
\usepackage{tikz}
\usepackage{dsfont} %Numeros reales, naturales...

%\graphicspath{{images/}}
\newcommand*{\QED}{\hfill\ensuremath{\square}}

%Estructura de ecuaciones
\setlength{\textwidth}{15cm} \setlength{\oddsidemargin}{5mm}
\setlength{\textheight}{23cm} \setlength{\topmargin}{-1cm}



\author{David García Curbelo}
\title{Ecuaciones}

\pagestyle{empty}


\def\R{\mathds{R}}
\def\Z{\mathds{Z}}
\def\N{\mathds{N}}

\def\sup{$^2$}


\begin{document}

    \setcounter{page}{1}
    \pagestyle{plain}
    \markright{Relación 4 Ecuaciones Diferenciales}

    \begin{center}
        {\large\bf{Relación 4 Ecuaciones diferenciales}} \\
        \bf{Parte 2}\\
        
    \end{center}

    \textbf{Ejercicio 1. } \\

    Nos piden encontrar la solución al sistema dado por 
    \begin{equation*}
        \left.
        \begin{aligned}
            x' + ty &= -1\\
            y' + x' &= -1\\
        \end{aligned}
        \right\}\quad \Rightarrow \quad
        \left.
        \begin{aligned}
            x' &= -ty -1\\
            y' &= ty + 1 - 1\\
        \end{aligned}
        \right\}\quad \Rightarrow \quad
        \left.
        \begin{aligned}
            x' &= -ty -1\\
            y' &= ty\\
        \end{aligned}
        \right\} 
    \end{equation*}
    Dicho sistema podemos expresar de forma $X'(t)=A(t)X(t) + B(t)$, donde vemos que cada uno de los elementos viene dado por

    \begin{equation*}
        X'(t) = 
        \begin{pmatrix}
            x' \\
            y' \\
        \end{pmatrix}\\
        A(t) = 
        \begin{pmatrix}
            0 & -t  \\
            0 & t  \\
        \end{pmatrix}\\
        X(t) = 
        \begin{pmatrix}
            x \\
            y \\
        \end{pmatrix}\\
        B(t) = 
        \begin{pmatrix}
            -1 \\
            0 \\
        \end{pmatrix}\\
    \end{equation*}
    Nos encontramos ante un sistema de ecuaciones diferenciales lineales de primer orden. Podemos ver que la segunda ecuación sólo tiene coeficientes
    en $y$,  ya que viene dada por $y' = ty$, la cual sabemos que tiene por solución
    $$y(t) = K\cdot e^{\frac{t^2}{2}}, \quad K\in \R$$
    Sustituyendo en la primera ecuación, la cual viene dada por $x' = -ty -1$, que sustituyendo la solución obtenida de la ecuación anterior obtenemos
    $$x'(t) = -tKe^{\frac{t^2}{2}} - 1$$
    Que integrando en ambos lados resulta
    $$x(t) = - Ke^{\frac{t^2}{2}} - t + C, \quad C\in \R$$
    y obtenemos así la solución general del sistema planteado, en función de un parametro $K\in\R$, el cual variará em función de la condición inicial.
    $$X(t) = \begin{pmatrix}
        Ke^{\frac{t^2}{2}} - t + C\\
        Ke^{\frac{t^2}{2}} \\
    \end{pmatrix}, \quad K\in \R$$
    
    Y ahora, para el cálculo de una matriz solución particular, bastará con sustituir los parámetros que tenemos para obtener dos soluciones distintas del problema
    inicial. Tomemos por ello
    \begin{equation*}
        x_1(t) = 
        \begin{pmatrix}
            -t \\
            0 \\
        \end{pmatrix}\\
        x_2(t) = 
        \begin{pmatrix}
            e^{\frac{t^2}{2}} - t \\
            e^{\frac{t^2}{2}} \\
        \end{pmatrix}
    \end{equation*}
    Donde hemos tomado para la primera solución $K=C=0$, y para la segunda $K=1, \thinspace C=0 $, y por ello podemos construir su matriz solución dada por:
    $$
        \Phi (t) = 
        \begin{pmatrix}
            -t & e^{\frac{t^2}{2}} - t \\
            0 & e^{\frac{t^2}{2}} \\           
        \end{pmatrix}
    $$
\end{document}