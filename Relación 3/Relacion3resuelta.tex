\documentclass[fleqn]{article}

%\pgfplotsset{compat=1.17}

\usepackage{mathexam}
\usepackage{amsmath}
\usepackage{amsfonts}
\usepackage{graphicx}
\usepackage{systeme}
\usepackage{microtype}
\usepackage{multirow}
\usepackage{pgfplots}
\usepackage{listings}
\usepackage{tikz}
\usepackage{dsfont} %Numeros reales, naturales...

%\graphicspath{{images/}}
\newcommand*{\QED}{\hfill\ensuremath{\square}}

%Estructura de ecuaciones
\setlength{\textwidth}{15cm} \setlength{\oddsidemargin}{5mm}
\setlength{\textheight}{23cm} \setlength{\topmargin}{-1cm}



\author{David García Curbelo}
\title{Ecuaciones}

\pagestyle{empty}


\def\R{\mathds{R}}
\def\sup{$^2$}


\begin{document}

    \setcounter{page}{1}
    \pagestyle{plain}
    \markright{Relación 3 Ecuaciones Diferenciales}

    \begin{center}
        {\large\bf{Relación 3 Ecuaciones diferenciales}} \\
        %\bf{David García Curbelo}\\
        
    \end{center}


    {\bf{Ejercicio 1.}} \\

    Es fácil ver que $F(x)$ es derivable. Por ello tomamos $F(x)=\int_0^1 e^{\theta x^2} \cos^2 (\theta) d\theta$,
    $$F'(x)=\frac{\partial}{\partial x}\int_0^1 e^{\theta x^2} \cos^2 (\theta) d\theta = 
    \int_0^1 \frac{\partial}{\partial x} e^{\theta x^2} \cos^2 (\theta) d\theta
    = \int_0^1 2x\theta e^{\theta x^2} \cos^2 (\theta) d\theta$$
    Sustituimos ahora el valor $x=0$ en $F'(x)$, y obtenemos
    $$F'(0)=\int_0^1 2 \cdot 0 \cdot \theta e^{\theta \cdot 0^2} \cos^2 (\theta) d\theta = 
    \int_0^1 0 d\theta = 0$$ \QED \\ \\

    {\bf{Ejercicio 2.}} \\

    Para comprobar que una ecuación sea exacta, debemos verificar primero que se cumple la condición de exactitud,
    es decir, ha de cumplirse la igualdad $\frac{\partial P(x,y)}{\partial y}=\frac{\partial Q(x,y)}{\partial x}$.
    Dada la ecuación $\frac{e^x}{y+e^x} + 2x + \frac{1}{y+e^x}y' = 0$ tomamos $P(x,y)=\frac{e^x}{y+e^x} + 2x$, 
    $Q(x,y)=\frac{1}{y+e^x}$. 
    $$\frac{\partial P(x,y)}{\partial y}=\frac{-e^x}{(y+e^x)^2}=\frac{\partial Q(x,y)}{\partial x}$$
    Como se cumple la condición de exactitud, la ecuación es exacta. \\
    
    Para el segundo apartado, vamos a encontrar la solución al problema de valores iniciales con condición inicial
    $y(0)=0$. Vemos que en este caso $Q(x,y)\neq 0, \forall (x,y) \in \Omega$, con $y+e^x \neq 0$. Como con nuestra
    condición inicial $Q \neq 0$ y sabemos que nuestro dominio es estrellado, podemos afirmar la existencia de potencial
    en todo el dominio. Procedemos a calcularlo:
    $$U(x,y)=\int P(x,y) dx +f(y)= \int (\frac{e^x}{y+e^x} + 2x)dx + f(y) = ln(y+e^x) + x^2 +f(y)$$
    Como $\frac{\partial U(x,y)}{\partial y}=Q(x,y)$, tenemos
    $$\frac{1}{y+e^x} + f'(y)=\frac{1}{y+e^x}$$
    y por tanto $f'(y)=0$, es decir $f(y)$ es constante, luego 
    $$U(x,y)=ln(y+e^x) + x^2+C$$
    Como $U$ se conserva a lo largo de todas las soluciones del dominio, (y tenemos que $y(0)=0$) podemos afirmar que
    $$U(x,y)=U(0,y(0))=U(0,0)$$
    $$ln(y+e^x) + x^2=0$$
    Despejando $y(x)$ obtenemos
    $$y(x)=e^{-x^2}-e^x, \forall x \in \mathds{R}$$
    \QED \\ \\

    {\bf{Ejercicio 3.}} \\

    Sabemos por hipótesis que tanto $P$ como $Q$ son ambas diferenciables. Por el teorema fundamental del cáclulo podemos obtener
    $$\frac{\partial U}{\partial x}=\frac{\partial }{\partial x} \int _0^x P(s,y) ds = P(x,y)$$
    $$\frac{\partial U}{\partial y}=\frac{\partial }{\partial y} \int _0^y Q(0,s) ds + \frac{\partial }{\partial y} \int _0^x P(s,y) ds =
    Q(0,y)+ \int _0^x \frac{\partial }{\partial y}P(s,y) ds$$
    Como por hipótesis sabemos que $\frac{\partial U}{\partial y}=\frac{\partial Q}{\partial x}$, entonces tenemos
    $$\frac{\partial U}{\partial y}=Q(0,y)+ \int _0^x \frac{\partial }{\partial x}Q(s,y) ds=Q(x,y)$$
    como queríamos probar. \QED \\ \\

    {\bf{Ejercicio 4.}} \\

    Nos encontramos ante un problema de valores iniciales, donde $P(x,y)=y-4x^3$ y $Q(x,y)=2y+x$, ambas continuas y diferenciables
    en todo su dominio ($\Omega = \mathds{R}^2$). Vemos que se cumple la condición de exactitud, ya que 
    $\frac{\partial P}{\partial y}=1=\frac{\partial Q}{\partial x}$, luego como se trata de un dominio estrellado y además 
    cumple la condición de exactitud, podemos afirmar la existencia de potencial. Además podemos afirmar que, como $Q(x_0, y(x_0))\neq 0$, 
    sabemos que existe un intervalo maximal en el que estará definida la función, y por lo tanto tendrá solución. Procedamos a calcularlo:
    $$U(x,y)=\int P(x,y) dx +f(y)= \int (y-4x^3)dx + f(y) = yx - x^4 + f(y)$$
    Como $\frac{\partial U(x,y)}{\partial y}=Q(x,y)$, tenemos
    $$x + f'(y)=2y+x$$
    Y por lo tanto $f(y)=y^2+C$ donde $C$ es una constante real. Obtenemos por tanto que el potencial queda de la forma
    $$U(x,y)= yx - x^4 + y^2 + C$$
    Aplicamos a continuación la condición inicial  $y(0)=-1$. Además como $U$ se conserva a lo largo de todas las soluciones del dominio,
    podemos afirmar que $$U(x,y)=U(0,y(0))=U(0,-1)$$
    $$yx - x^4 + y^2=1$$
    $$y(x)=\frac{-x \pm \sqrt{x^2 +4(x^4+1)}}{2}$$
    Pero como sabemos que $y(0)=-1$, concluimos que la solución viene dada por 
    $$y(x)=-\frac{x + \sqrt{x^2 +4(x^4+1)}}{2}, \forall x \in \mathds{R}$$
    \QED \\ \\

    {\bf{Ejercicio 5.}} \\

    Nos encontramos ante otro problema de valores iniciales como el planteado en el ejercicio anterior, donde $P(x,y)=e^{x-1} + y$ y $Q(x,y)=x$, 
    ambas continuas y diferenciables en todo su dominio ($\Omega = \mathds{R}^2$). Vemos que se cumple la condición de exactitud, ya que 
    $\frac{\partial P}{\partial y}=1=\frac{\partial Q}{\partial x}$, luego como se trata de un dominio estrellado y además 
    cumple la condición de exactitud, podemos afirmar la existencia de potencial. Procedamos a calcularlo:
    $$U(x,y)=\int P(x,y) dx +f(y)= \int (e^{x-1} + y)dx + f(y) = e^{x-1} + yx + f(y)$$
    Como $\frac{\partial U(x,y)}{\partial y}=Q(x,y)$, tenemos
    $$x + f'(y)=x$$
    y por tanto $f'(y)=0$, es decir $f(y)$ es constante, luego
    $$U(x,y)=e^{x-1} + yx + C$$
    Como $U$ se conserva a lo largo de todas las soluciones del dominio, (y tenemos que $y(1)=0$) podemos afirmar que
    $$U(x,y)=U(1,y(1))=U(1,0)$$
    $$e^{x-1} + yx = 1$$
    Despejando $y(x)$ obtenemos
    $$y(x)=\frac{1-e^{x-1}}{x}, \forall x \in \mathds{R}-(0)$$ 
    \QED \\ \\

    {\bf{Ejercicio 6.}} \\

    Supongamos existe dicho factor integrante $\mu(x,y)=m(x+2y)$ para el que se cumple por tanto la condición 
    $\frac{\partial (\mu P)}{\partial y}=\frac{\partial (\mu Q)}{\partial x}$, es decir, la condición de exactitud para
    $\mu (P + Qy')=0$. Desarrollemos cada uno de los elementos de la igualdad:
    $$\frac{\partial (\mu P)}{\partial y}=\frac{\partial (\mu Q)}{\partial x}$$
    $$P\frac{\partial \mu}{\partial y} + \mu \frac{\partial P}{\partial y}=Q\frac{\partial \mu}{\partial x} + \mu \frac{\partial Q}{\partial x}$$
    $$P\frac{\partial \mu}{\partial y}-Q\frac{\partial \mu}{\partial x}=\mu (\frac{\partial Q}{\partial x}-\frac{\partial P}{\partial y})$$
    $$P(2m)-Q(m)=m(x+2y) (\frac{\partial Q}{\partial x}-\frac{\partial P}{\partial y})$$
    $$2P-Q=(x+2y) (\frac{\partial Q}{\partial x}-\frac{\partial P}{\partial y})$$
    $$\frac{2P-Q}{\frac{\partial Q}{\partial x}-\frac{\partial P}{\partial y}}=x+2y $$
    Donde suponemos que $\frac{\partial Q}{\partial x}-\frac{\partial P}{\partial y} \neq 0$ (es decir, que no se cumpla la condición de exactitud,
    ya que si se cumpliera no nos haría falta ningún factor integrante). Y por lo tanto tenemos la condición para la existencia de dicho potencial.
    Vemos que las constantes que multiplican en el factor integrante no son relevantes, ya que no influyen en las condiciones del mismo 
    para su existencia.\QED \\ \\
    
    \newpage

    {\bf{Ejercicio 7.}} \\

    Para este ejercicio, antes de buscar el factor integrante, comprobemos que ciertamente no se cumple la condición de exactitud. Tomamos 
    $P(t,x)=2t+t^2x$, $Q(t,x)=1$. Calculamos sus parciales
    $$\frac{\partial Q}{\partial t}=0 \neq t^2=\frac{\partial P}{\partial x}$$
    y por lo tanto no es una ecuación exacta. Procedemos al cálculo del factor integrante dado por $m(t)$. Sabemos por el ejercicio anterior que
    este factor ha de cumplir
    $$P\frac{\partial m(t)}{\partial x}-Q\frac{\partial m(t)}{\partial t}=m(t) \left(\frac{\partial Q}{\partial t}-\frac{\partial P}{\partial x}\right)$$
    $$P\cdot 0-1 \cdot m'(t)=m(t) \left(0-t^2\right)$$
    $$m'(t)=t^2m(t)$$
    Obtenemos una ecuación diferencial de primer orden cuya solución viene dada por $m(t)=e^{\frac{t^3}{3}}$, con lo que obtenemos el factor integrante
    buscado. Comprobemos que se cumple la condición de exactitud en $m(t)\left(2t+t^2x+x'\right)=0$
    $$\frac{\partial (m(t)Q)}{\partial t}=t^2e^{\frac{t^3}{3}}=\frac{\partial m(t)P}{\partial x}$$
    luego tenemos el factor integrante que buscábamos.\QED \\ \\

    {\bf{Ejercicio 8.}} \\

    Se trata de un ejercicio muy similar al anterior, en el que ahora el factor integrante sólo depende de $x$. Para comprobar su existencia basta ver si
    la ecuación siguiente 
    $$P\frac{\partial m(x)}{\partial y}-Q\frac{\partial m(x)}{\partial x}=m(x) \left(\frac{\partial Q}{\partial x}-\frac{\partial P}{\partial y}\right)$$
    admite una solución para $m(x)$, donde $P(x,y)=y^2$ y $Q(x,y)=e^x$ y para los cuales no se cumple la condición de exactitud. Sustituimos en la 
    igualdad anterior y obtenemos
    $$y^2 \cdot 0-e^xm'(x)=m(x)\left(e^x-2y\right)$$
    $$m'(x)=m(x)\left(\frac{2y}{e^x}-1\right)$$
    Podemos ver que la solución va a tener dependencia en la variable $y$, por lo tanto podemos afirmar que no existe 
    factor integrante en estas condiciones.\QED \\ \\
    
    \newpage

    {\bf{Ejercicio 9.}} \\

    Veamos si admite potencial. Para ello, vemos que el dominio en que está definida nuestra función $F$ es un dominio estrellado. Si además
    comprobamos la condición de exactitud podremos afirmar la existencia de dicho potencial. 
    $$\frac{\partial P}{\partial y} = \frac{-2x}{y^2} = \frac{\partial Q}{\partial x}$$
    Luego se cumple la condición de exactitud, y por tanto afirmamos la existencia de potencial $U$.

    Procedamos a continuación al cálculo del trabajo a lo largo de la curva $\gamma (\theta)=(\cos\theta, 1+\sin\theta)$, con $\theta \in [0,\pi]$.
    Para ello, consideremos la siguiente integral
    $$T=\int_0^{\pi} \langle F\left(\gamma(\theta)\right), \gamma'(\theta)\rangle d\theta = 
    \int_0^{\pi} \langle \left(\frac{2\cos\theta}{1+\sin\theta}, \frac{-\cos^2\theta}{(1+\sin\theta})^2\right), (-\sin\theta, \cos\theta) \rangle d\theta = $$
    $$\int_0^{\pi} \frac{-2\cos\theta \sin\theta}{1+\sin\theta}d\theta - \int_0^{\pi} \frac{\cos^3\theta}{(1+\sin\theta)^2}d\theta = 0-0 = 0$$

    Mediante procesos elementales de integración llegamos a la conclusión de que ambas integrales tienen valor 0, y por tanto el trabajo es nulo
    en el recorrido a lo largo de la curva antes mencionada. \QED \\ \\

    {\bf{Ejercicio 10.}} \\

    Como el campo de fuerzas $F$ está definido en un dominio estrellado $\R^2$, si cumple la condición de exactitud, podemos afirmar la existencia
    de potencial $U:\R^2 \rightarrow \R$ tal que $F=\nabla U$. Forzamos a continuación que se cumpla la condición de exactitud, es decir
    $$\frac{\partial F_1}{\partial y} = 1 = \frac{\partial F_2}{\partial x}$$
    
    Integrando $F_1$ con respecto a $y$ (como haríamos en otro caso con P) obtenemos que $F_1(x,y)=y+ f(x)$. Nosotros tenemo ya que
    $F_2(x,y)=x+y$ Calculemos a continuación el potencial asociado
    $$\int F_2(x,y) dy = xy + \frac{y^2}{2} + g(x)$$
    
    Como $\frac{\partial U}{\partial y} = F_1$ obtenemos que $g(x)=\int f(x)dx + C$. La expresión de nuestro potencial queda
    $$U(x,y)= xy + \frac{y^2}{2} \int f(x)dx$$
    Para una cierta función $f(x)$

    Calculamos ahora el trabajo a lo largo de la curva $\alpha (s) = (\cos s, 2+ \sin s)$, con $s \in \left[\frac{-\pi}{2},\frac{\pi}{2}\right]$
    $$T=\int _a^b \langle F(\alpha(s)), \alpha'(s) \rangle ds = U(\alpha(b))-U(\alpha(a)) =
    $$$$U((0,3))-U((0,1)) = \frac{3^2}{2} + \int f(0)dx - \left(\frac{1^2}{2} + \int f(0)dx\right)=4$$
    Con lo que tenemos el trabajo realizado al recorrer la trayectoria de la curva antes mencionada. \QED


\end{document}